
\usepackage{lmodern}

\usepackage[utf8]{inputenc}
\usepackage[T1]{fontenc}
\usepackage{url}

\usepackage{xspace}
\newcommand{\ie}[0]{{\em i.e.},\xspace}
\newcommand{\vs}[0]{{\em vs.}\xspace}
\newcommand{\eg}[0]{{\em e.g.},\xspace}
\newcommand{\etal}[0]{{\em et al.}\xspace}
\newcommand{\wrt}[0]{{\em w.r.t.}\xspace}
\newcommand{\aka}[0]{{\em a.k.a.}\xspace}
\newcommand{\via}[0]{{\em via}\xspace}
\newcommand{\mad}[0]{Madeus\xspace}
\newcommand{\implem}[0]{PoC\xspace}
\newcommand{\llc}{L\textsuperscript{2}C\xspace}
\newcommand{\net}[0]{\emph{internal-net}\xspace}

\usepackage{caption}

\usepackage{hyperref}
\usepackage{url}

\usepackage[textwidth=17mm]{todonotes}
\newcommand{\customtodo}[4]{
        \todo[color=#2,inline,size=\small]{
                \ifx&#3&
                        \textbf{#1} #4
                \else
                        \textbf{#1$\Rightarrow$#3} #4
                \fi
        }
}
\newcommand{\MC}[2][]{\customtodo{MC}{red!50}{#1}{#2}}
\newcommand{\HC}[2][]{\customtodo{HC}{red!20}{#1}{#2}}
\newcommand{\CP}[2][]{\customtodo{CP}{blue!20}{#1}{#2}}
\newcommand{\DP}[2][]{\customtodo{DP}{blue!50}{#1}{#2}}
\newcommand{\CS}[2][]{\customtodo{CS}{green!20}{#1}{#2}}
\usepackage{amsmath,amssymb,amsfonts}
\usepackage{mathtools}
\usepackage{subfig}
\usepackage{pifont}
\usepackage{wasysym}
\usepackage{booktabs} % For formal tables

\usepackage{tikz}
\usetikzlibrary{shapes.geometric, arrows}
\tikzstyle{place} = [rectangle, rounded corners, minimum width=2cm, minimum height=1cm,text centered, draw=black, fill=red!40]
\tikzstyle{leaving} = [rectangle, rounded corners, minimum width=2cm, minimum height=1cm,text centered, draw=black, fill=red!20]
\tikzstyle{provide_start} = [rectangle, rounded corners, minimum width=2cm, minimum height=1cm,text centered, draw=black, fill=orange!40]
\tikzstyle{provide_stop} = [rectangle, rounded corners, minimum width=2cm, minimum height=1cm,text centered, draw=black, fill=blue!20]
\tikzstyle{use_start} = [rectangle, rounded corners, minimum width=2cm, minimum height=1cm,text centered, draw=black, fill=orange!20]
\tikzstyle{use_stop} = [rectangle, rounded corners, minimum width=2cm, minimum height=1cm,text centered, draw=black, fill=blue!40]
\tikzstyle{end} = [rectangle, rounded corners, minimum width=2cm, minimum height=1cm,text centered, draw=black, fill=gray!20]
\tikzstyle{beginning} = [rectangle, rounded corners, minimum width=2cm, minimum height=1cm,text centered, draw=black, fill=gray!40]
\tikzstyle{control} = [rectangle, rounded corners, minimum width=2cm, minimum height=1cm,text centered, draw=black, fill=orange!30]
\tikzstyle{arrow} = [thick,->,>=stealth]

% lstlisting python + colors
\usepackage{color}
\usepackage[procnames]{listings}
\definecolor{keywords}{RGB}{255,0,90}
\definecolor{comments}{RGB}{0,0,113}
\definecolor{red}{RGB}{160,0,0}
\definecolor{green}{RGB}{0,150,0}
\definecolor{backcolour}{rgb}{0.95,0.95,0.95}
\DeclareCaptionFont{white}{\color{white}}
\DeclareCaptionFormat{listing}{\colorbox{gray}{\parbox{\linewidth}{#1#2#3}}}
\captionsetup[lstlisting]{format=listing,labelfont=white,textfont=white}
\lstset{language=Python,
        numbers=left,
        backgroundcolor=\color{backcolour},
        numbersep=5pt,
        keywordstyle=\color{keywords},
        commentstyle=\color{comments},
        stringstyle=\color{red},
        showstringspaces=false,
        identifierstyle=\color{green},
        procnamekeys={def,class},
        basicstyle=\scriptsize\ttfamily,
        numberstyle=\tiny\color{gray},
%       frame=single
}

