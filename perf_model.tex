In this section, we present Madeus' performance model. Its goal is,
given the execution time of all the transitions of all the components
in an assembly, to predict the total execution time of the deployment.
Intuitively, we automatically model the execution flow of a Madeus
assembly based on its formal semantics. This is done by generating graph
representations of the execution flow of each Madeus component in the
assembly and then connecting them together according to their dependencies.
Each component's graph is a directed acyclic graph (DAG) because there can
be no cycles in a Madeus component's internal-net. Once connected, and
under some minor restrictions essentially preventing non-determinism, the
graph is still a DAG. By looking for the longest path between a source
vertex representing the beginning of the deployment process and a sink
vertex representing its end, we can predict the total deployment time.
This can be done in $\mathcal{O}(|V|+|E|)$ where $V$ is the set of
vertices of the graph and $E$ is the set of edges of the graph. Note that
$|V| = \mathcal{O}(pl+tr+po)$ and $|E| = \mathcal{O}(tr+po+dp)$ where $pl$
is the number of places in the whole assembly, $tr$ is the number of
transitions, $po$ is the number of ports and $dp$ is the number of
dependencies. Therefore, the complexity of estimating the running time of
a Madeus deployment is $\mathcal{O}(pl+tr+po+dp)$.
% By generating a graph modelling the execution flow of a Madeus assembly,
% capturing both intra-component and inter-component dependencies, we
% reduce this problem to finding the longest path in a DAG.

\subsection{Notations}

Recall that given a component, we denote its set of places $\Pi$ and its
set of transitions $\Theta$. In the following, we will consider for the
sake of simplicity that the transitions go directly from a place to
another place instead of from an output dock to an input dock.
Hence, $\Theta$ is a multiset which elements are couples of places. We
can obtain the place corresponding to each dock by using the $place$
function of the component.

In order to predict the total deployment time, we need to know the
execution time of each individual transition. In the following, we
consider the \emph{time function} $time\,:\,\Theta\rightarrow\mathbb{R}^{+}$
associating a running time to each transition (taken as input).

We also consider the following two functions:
\begin{itemize}
\item the \emph{group entrance function} $g_{in}\,:\,G\rightarrow\mathcal{P}\left(\Pi\right)\,g\mapsto\left\{ \pi\,\mid\,\pi\in g\land\exists\pi_{b}\,:\,\left(\pi_{b}\not\in g\land\left(\pi_{b},\pi\right)\in\Theta\right)\right\} $
(the result of $g_{in}$ is called the set of \emph{entrance places}
of the group)
\item the \emph{group exit function} $g_{out}\,:\,G\rightarrow\mathcal{P}\left(\Pi\right)g\mapsto\left\{ \pi\,\mid\,\pi\in g\land\exists\pi_{a}\,:\,\left(\pi_{a}\not\in g\land\left(\pi,\pi_{a}\right)\in\Theta\right)\right\} $
(the result of $g_{out}$ is called the set of \emph{exit places}
of the group)
\end{itemize}

Recall that an assembly is a tuple $\left(C,L_{P},L_{D},ebl\right)$. In
the following, we consider that
$C=\bigcup_{i=1}^{n}\left\{ \left(\Pi_{i}\dots,\left(B_{D_{p}}\right)_{i}\right)\right\} $.
For each notation $X$ specific to a component, we denote $X_{all}$ the
union (in the case of an assembly) or the extension (in the case of
a function) for all components. For instance:
\begin{itemize}
\item $\Pi_{all}=\bigcup_{i=1}^{n}\Pi_{i}$ (set of all \emph{places} in
the assembly)
\item $\left(time\right)_{all}\,:\,\Theta_{all}\rightarrow\mathbb{R}^{+}$
(function giving the \emph{running time} of each transition) with:
$\left(time\right)_{all}\left(x\right)=\left(time\right)_{i}\left(x\right)$
if $x\in\Theta_{i}$ 
\end{itemize}
The execution flow graph is an oriented weighted graph \emph{$\left(V,A\right)$}
where $V$ is the set of vertices and $A$ is the multiset of weighted
arcs with elements in $V\times V\times\mathbb{R}^{+}$. We define
$V$ an $A$ in the following.

\subsection{Vertices}

For each place, we associate two vertices: one representing the place
itself and one representing the action of a token leaving the place.
\[
V_{\Pi}=\bigcup_{\pi\in\Pi_{all}}\left\{ v_\pi^\text{place},v_\pi^\text{leaving}\right\} 
\]

For each transition, we associate two vertices: one representing the
beginning of the transition and one representing its end. 
\[
V_{\Theta}=\bigcup_{\theta\in\Theta_{all}}\left\{ v_\theta^\text{beginning},v_\theta^\text{end}\right\} 
\]

For each data (use or provide) port we associate one vertex representing
its activation. 
\[
V_{data}=\bigcup_{u\in\left(D_{u}\right)_{all}\cup\left(D_{p}\right)_{all}}\left\{ v_u^\text{start}\right\} 
\]

For each service (use or provide) port we associate two vertices:
one representing its activation and one its deactivation. 
\[
V_{service}=\bigcup_{p\in\left(P_{u}\right)_{all}\cup\left(P_{p}\right)_{all}}\left\{v_p^\text{start},v_p^\text{stop}\right\} 
\]

Finally, we define $V$ as the union of all these, plus one source
and one sink vertices. 
\[
V=V_{\Pi}\cup V_{\Theta}\cup V_{data}\cup V_{service}\cup\left\{ v^\text{source},v^\text{sink}\right\} 
\]


\subsection{Arcs}

In the performance graph, arcs represent time constraints: the event represented
by the destination vertex must happen after the one represented by the source
vertex, at least $w$ seconds apart where $w$ is the weight of the arc. In
practice, the weight of all of the arcs except those corresponding to the
transitions is 0. The weights of the latter are the execution times of the
transitions.

For each place $\pi$ we associate one arc going from $v_\pi^\text{place}$ to
$v_\pi^\text{leaving}$. This represents the fact that a token may leave $\pi$
only after it has entered it.
\[
A_{\Pi}=\bigcup_{\pi\in\Pi_{all}}\left\{ \left(v_\pi^\text{place},v_\pi^\text{leaving},0\right)\right\} 
\]

For each transition $\theta$ going from $\pi_s$ to $\pi_d$, we associate
three arcs. The first between $v_\theta^\text{beginning}$ and $v_\theta^\text{end}$
represents the fact that the end of $\theta$ happens at the time of the
start of $\theta$ plus its duration. The second between $v_{\pi_{s}}^\text{leaving}$
and $v_\theta^\text{beginning}$ represents the fact that $\theta$ may only
happen after a token leaved $\pi_s$. The third one between $v_\theta^\text{end}$ and
$v_{\pi_{d}}^\text{place}$ represents the fact that a token may enter $\pi_d$ only
after $\theta$ has finished.
\begin{align*}
A_{\Theta}=\bigcup_{\theta=\left(\pi_{s},\pi_{d}\right)\in\Theta_{all}} & \left\{ \left(v_\theta^\text{beginning},v_\theta^\text{end},time_{all}\left(\theta\right)\right),\right.\\
 & \left(v_{\pi_{s}}^\text{leaving},v_\theta^\text{beginning},0\right),\\
 & \left. \left(v_\theta^\text{end},v_{\pi_{d}}^\text{place},0\right)\right\}
\end{align*}

For each binding between a data use port $p$ and a transition $\theta$ we
associate one arc going from $v_p^\text{start}$ to $v_\theta^\text{beginning}$.
This represents the fact that the transition may only begin after the port has
received data.
\[
A_{D_{u}}=\bigcup_{\left(p,\theta\right)\in\left(B_{D_{u}}\right)_{all}}\left\{ \left(v_p^\text{start},v_\theta^\text{beginning},0\right)\right\} 
\]

For each binding between a data provide port $p$ and a place $\pi$ we associate
one arc going from $v_\pi^\text{place}$ to $v_p^\text{start}$. This represents
the fact that the data may only be provided after the place $\pi$ has been
reached by a token.
\[
A_{D_{p}}=\bigcup_{\left(p,\pi\right)\in\left(B_{D_{p}}\right)_{all}}\left\{ \left(v_\pi^\text{place},v_p^\text{start},0\right)\right\} 
\]

For each binding between a service use port $p$ and a transition $\theta$ we
associate two arcs. The first from $v_p^\text{start}$ to
$v_\theta^\text{beginning}$ represents the fact that $\theta$ can only start
once the service is provided. The second from $v_\theta^\text{end}$ to
$v_p^\text{stop}$ represents the fact that the service may only stop being
provided after $\theta$ has finished.
\[
A_{P_{u}}=\bigcup_{\left(p,\theta\right)\in\left(B_{P_{u}}\right)_{all}}\left\{ \left(v_p^\text{start},v_\theta^\text{beginning},0\right),\left(v_\theta^\text{end},v_p^\text{stop},0\right)\right\} 
\]

For each binding between a service provide port $p$ and a group $g$ we
associate two sets of arcs. The first set of transitions going from
$v_\pi^\text{place}$ to $v_p^\text{start}$ for each place $\pi$ in the
entrance of $g$ represents the fact that a token must enter the group
before the service becomes active. The second set of transitions going
from $v_p^\text{stop}$ to $v_\pi^\text{leaving}$ for each place
$\pi$ in the exit of $g$ represents the fact that the service must not
be in use anymore before all tokens can leave the group.
\begin{align*}
A_{P_{p}}=\bigcup_{\left(p,g\right)\in\left(B_{P_{p}}\right)_{all}} & \left( \bigcup_{\pi\in\left(g_{in}\right)_{all}\left(g\right)}\left\{ \left(v_\pi^\text{place},v_p^\text{start},0\right)\right\} \cup \right. \\
 & \left. \bigcup_{\pi\in\left(g_{out}\right)_{all}\left(g\right)}\left\{ \left(v_p^\text{stop},v_\pi^\text{leaving},0\right)\right\} \right)
\end{align*}

For each connection between data provide port $p$ and data use port $u$
in the assembly, we associate one arc going from $v_p^\text{start}$ to
$v_u^\text{start}$. This represents the fact that the data may only be
used after it has been provided.
\[
A_{L_{D}}=\bigcup_{\left(p,u\right)\in L_{D}}\left\{ \left(v_p^\text{start},v_u^\text{start},0\right)\right\} 
\]

For each connection between service provide port $p$ and service use port
$u$ in the assembly, we associate two arcs. The first from
$v_p^\text{start}$ to $v_u^\text{start}$ represents the fact that the
service may only be used after it has started being provided.
The second from $v_u^\text{stop}$ to $v_p^\text{stop}$ represents the fact
that the service may stop being provided only after it has stopped being
used.
\[
A_{L_{P}}=\bigcup_{\left(p,u\right)\in L_{P}}\left\{ \left(v_p^\text{start},v_u^\text{start},0\right),\left(v_u^\text{stop},v_p^\text{stop},0\right)\right\} 
\]

For each initial place $\pi$ we associate one arc going from $v^\text{source}$
 to $v_\pi^\text{place}$, representing the fact that a token is placed in each
 initial place at the very beginning.
\[
A_{I}=\bigcup_{\pi\in I_{all}}\left\{ \left(v^\text{source},v_\pi^\text{place},0\right)\right\} 
\]

In addition to the set of all initial places $I_{all}$, we define
the set of all final places $F_{all}$ as the set of places which
do not have any outgoing transition. Formally,
$F_{all}=\left\{ \pi\,\mid\,\pi\in\Pi_{all}\land\lnot\left(\exists\pi_{a}\in\Pi_{all}\,:\,\left(\pi,\pi_{a}\right)\in\Theta_{all}\right)\right\} $.
Then, for each final place $\pi$ we associate one arc going from
$v_\pi^\text{place}$ to $v^\text{sink}$, representing the fact that the
deployment is over only after all components have reached their final places
(\ie when no more transition can be executed).
\[
A_{F}=\left\{ v_\pi^\text{place},v^\text{sink},0\right\} 
\]

Finally, we define $A$ as the union of all of these. 
\[
A=A_\Pi\cup A_{\Theta}\cup A_{D_{u}}\cup A_{D_{p}}\cup A_{P_{u}}\cup A_{P_{p}}\cup A_{L_{D}}\cup A_{L_{P}}\cup A_{I}\cup A_{F}
\]


\subsection{Time prediction}

We define the time prediction of the execution of the Madeus assembly
to be the length of the longest path between the \emph{source} vertex
and the \emph{sink} vertex in the graph $\left(V,A\right)$. This
path exists as long as the internal-nets of all the components are
individually connected, and is finite because there can be no loop
in internal-nets and the execution times considered for the transitions
are finite.


\subsection{Restrictions}

TODO
