\mad is a new component-based model specifically designed for
distributed software commissioning procedures. By adapting composition
mechanisms and combining them to the notion of control components \mad
enhances both the separation of concerns and the efficiency of
commissionings compared to the related work. In this paper the \mad
model has been extensively presented from both the theoretical and
experimental viewpoint.

First, after a detailed study of the related work, an overview of \mad
was given. Second, the formalization of \mad was presented such that
the theoretical model is streamlined compared to our previous
publications. Third, a performance prediction model of \mad was
studied and consists in the transformation of a \mad assembly to a
directed acyclic graph (DAG) that represents the execution flow of the
assembly. Fourth the prototype of \mad was evaluated on three
synthetic benchmarks and compared to the expected performance
predicted by the theoretical model. Results have shown a very low
overhead introduced by the prototype as well as an accurate prediction
of the performance model. Finally, contributions of \mad regarding the
separation of concerns and the efficiency has been evaluated and
discussed on a real complex use-case, the commissioning procedure of
OpenStack. Results have been extensively studied and have shown that
\mad outperforms both \ansible and \aeolus in terms of efficiency.

As future work we first plan a generalization of \mad to dynamic
reconfiguration of distributed software. Indeed, once commissioned
distributed software may adapt dynamically because of faults, or to
optimize some metrics (\eg energy, efficiency), or to adapt their
services to dynamic requirements (\eg smart cities). We think that
when reconfiguration decisions have to be taken, especially for
critical systems, the duration of the reconfiguration have to be taken
into account. Thus, generalizing both \mad and the performance
prediction models of Section~\ref{sec:perf_model} to dynamic
reconfiguration is an interesting contribution.

Second, \mad is currently not equiped with mechanisms to handle faults
during the commissioning procedure. Typically the rule $\terminaction$
do not consider the possibility of a failure during the action
$\alpha \in \exec$. We would like to equip \mad with if/else
statements or switches as well as rollback mechanisms. Moreover we
would like to have formal guarantees in case of faults, for instance
by using game theory or stochastic formal methods.

Finally, we would like to study the semi-automatic (thanks to a very
light DSL) or automatic (if possible) inference of dependencies from
\ansible playbooks such that the \mad assemblies can be entirely or
partially generated for the user.

  