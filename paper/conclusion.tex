\mad is a new component-based model specifically designed for
distributed software commissioning procedures. By adapting composition
mechanisms and combining them with the notion of control components,
\mad enhances both the separation of concerns and the efficiency of
commissioning compared to previous solutions. In this paper, the \mad
model has been extensively presented from both the theoretical and
experimental perspectives.

First, after a detailed study of the related work, an overview of \mad
was given. Second, the formalization of \mad was presented, with a
streamlined theoretical model compared to our previous
publications. Third, a performance prediction model of \mad was
studied, based on the transformation of a \mad assembly into a
directed graph that represents the execution flow of the
assembly. Fourth, the prototype of \mad was evaluated on three
synthetic benchmarks and compared to the expected performance
predicted by the theoretical model. Results have shown that the
overhead introduced by the prototype is very low, and that the
prediction of the performance model is accurate. Finally, the benefits
of \mad regarding separation of concerns and efficiency have been
evaluated and discussed on a real complex use case: the commissioning
procedure of OpenStack. Results have been extensively studied and have
shown that
\mad outperforms both \ansible and \aeolus in terms of efficiency, and that a 
projection on the traces of the OpenStack CI could save 34 hours of computation a 
day.

As future work, first, \mad is currently not equipped with mechanisms to handle faults during the commissioning procedure. Notably, the rule $\terminaction$
does not consider the possibility of a failure during the action
$\alpha \in \exec$. We would like to equip \mad with if/else statements or switches as well as rollback mechanisms. Moreover, we would like to provide formal guarantees in the presence of faults, for instance by using game theory or stochastic formal methods.

Second, we wish to study semi-automatic (\eg using a light DSL) or possibly automatic inference of dependencies from \ansible playbooks, so that \mad assemblies may be entirely or partially generated for the user.

Finally, we have already generalized \mad to perform dynamic reconfiguration of distributed software~\cite{ccgridmaverick}. Indeed, once commissioned, distributed software may need to adapt dynamically in order to respond to faults, to optimize some metrics (\eg energy, efficiency), or to adapt the services to dynamic requirements (\eg smart cities). When these reconfiguration decisions are taken, especially for critical systems, the duration of the reconfiguration should be taken into account. Thus, generalizing \mad and the performance prediction model of Section~\ref{sec:perf_model} to dynamic reconfiguration is an important contribution.

  
