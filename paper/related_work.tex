% 1. introduce metrics of RW
% - soft. eng., composition, code reuse, sep of concerns
% - parallelism levels
% 2. RW analyze
% - each tool
% - discussion and table

%----------------------------
\subsection{Related work metrics}
%----------------------------

% 1. software engineering
% - modules/components
% - tasks
% - ports
% 2. parallelism
% - node
% - component
% - task
% 3. formalism
% - formal model
% - performance model

%----------------------------
\subsection{Related work}
%----------------------------

% Shell-scripts
\paragraph{Shell scripts}
The traditional way operators automate software commissioning is by transcribing
the actions and configurations from READMEs and tutorials into a sequence of
commands in \shell scripts.
On the one hand, those scripts are written with low-level imperative languages,
and good programming skills, it is possible to express complex workflows (\eg
idempotency, parallelism, remote actions using \textsc{SSH}).
For instance, parallelism can be managed by combining command execution in the
background (\eg using the control operator \& in \bash) and synchronisation
commands like \emph{wait}.
On the other hand, as the system grows, any custom script becomes error-prone,
unpredictable, hard to understand and to maintain.
\shell scripts lack of software engineering aspects and there is no framework to
express modules nor ports. As a consequence, \shell scripts are not adapted for
distributed software commissioning.
% devstack is a set of shell script to deploy OpenStack on a single machine

The following tools bring features from software engineering to tackle these
issues. Rather than manually writing imperative instructions describing the
whole process, an operator composes her distributed software by providing a
description of the software architecture (\ie the modules with their
connexions). Then, the configuration management tool is in charge to bring the
systems to the desired state.

% Ansible
\paragraph{Ansible}
For the DevOps that are used to shell-scripts, \ansible has become a popular
configuration management tool since it relies on a simple syntax written in YAML
and does not require agents on administrated servers. Tasks are indeed managed
using only \textsc{SSH} and \textsc{Python} which are commonly installed on
every Linux distributions.
In comparison, similar tools like \chef, \puppet or \cfengine not only require
at least a slight understanding of Ruby or a custom language to get started, but
they are built on an agent-based architecture and requires prior agent
commissioning on remote hosts.

\ansible features software engineering aspects like separation of concerns by
defining the notion of \emph{roles} which can be seen as software components.
Each role contains a set of files that describe a sequences of tasks. To define
a composition, a specific file called an \ansible \emph{playbook} is used for
mapping the desired roles to the groups of nodes they will be applied to. Those
groups of nodes are defined in a separate file called the \emph{inventory}. When
\ansible is triggered, roles and their related tasks are sequentially executed
to the associated groups of nodes.  While tasks declarations are indeed manages
sequentially, each task is executed in parallel when mapped to multiple remote
hosts.

Typically, an operator who wants to commission an \apache web server and a
\mysql database would download two roles from Ansible Galaxy and register them
in an playbook. Since \ansible triggers roles in a sequential manner, if the
operator is not aware that the database must be commissioned before the web
server, she could make a mistake in the order of the roles she declared. This
aspect breaks the separation of concerns since the operator has to understand
role's internals to determine their order.

% Tasks are declared in an imperative way, however, Ansible relies heavily on
% declarative modules, most of which ensure task idempotency (operations are run
% once even if called multiple times).

% Deployware
\paragraph{Deployware}
\citeauthor{flissi2008ccgrid} proposed \deployware (DW) as a research effort to
deploy distributed software in the context of Grid
computing~\cite{flissi2008ccgrid}. Its implementation is based on the Fractal
component model. A component is called a \emph{Personality} and is associated to
a fixed set of commissioning actions (\ie install, configure, start, manage,
stop, unconfigure and uninstall).  Each action describes a sequence of tasks,
written with a specific high-level language that uses pre-defined instructions
(\eg execute command, copy a file).

While there is no notion of component ports, it is possible to express
dependencies between components to initiate automatic coordination. For
instance, when the operator triggers the action "install" on a component, the
same action is triggered recursively to its dependencies.

% Kubernetes
\paragraph{Kubernetes}
Initiated by Google, \kubernetes (K8S) is a popular
framework to commission distributed software in the form of micro-services that
are packaged as Docker containers.

A software component in \kubernetes is thus defined as Docker container. Such
components have no port to manage coordination, and its internals is fixed since
containers can only be started and stopped. As a consequence, the commissioning
process is error-prone. For instance, a web server can be started before the
required database and thus fails.
% depend_on exists but limited to "container is started"
% synchronization is possible but must be managed by internal script and not managed by k8s
% good soc, download containers and run k8s deploy but without synchonization,
    % commissioning is error-prone

% Juju
\paragraph{Juju}
Canonical has developed their own software commissioning solution: \juju
(\url{https://jujucharms.com/}) which aims at commissioning any kind of
application on top of different cloud providers (\eg AWS, OpenStack), and on
multiple types of resources (container, VM or bare-metal).

Its concepts are close to component models. Software modules are packaged as
\juju \emph{charms} which describe the software commissioning steps through a
set of scripts called \emph{hooks}. Operators define their composition in a
specific file called \emph{bundle} in which they declare the desired charms with
their \emph{relations}. A relation is an element declared between two charms and
used for component synchronization (by triggering hooks) and data sharing at
runtime, similarly to component ports.
% good soc, download charms and run juju deploy

% Aeolus
\paragraph{Aeolus}
\citeauthor{dicosmo2014ic} proposed \aeolus: a formal component model for the
cloud~\cite{dicosmo2014ic}. Their component model captures the internal states
of a component commissioning process thanks to a finite state machine. Each
state can be connected to use, provide, or conflict ports to declare
dependencies between the commissioning steps of different components. Hence,
such ports enable the coordination of the global deployment process.

Based on the above commissioning tools, a closer study of their different
features can be used as a guide in formulating a framework to compare them. In
the following, we discuss the several aspects regarding performance and software
engineering that are desirable. These aspects are summarized in
\cref{tab:comparison}.

\begin{table}[tp]
  \centering
  \small
  

% \begin{tabular}{cl|cccccc}
%     \toprule
%     & & Shell & Ansible & DW & K8S & Juju & Aeolus \\
%     \midrule
%     %\rowcolor{gray!15}
%     \multirow{3}{*}{\STAB{\rotatebox[origin=c]{90}{Parallel}}}
%     & node    & \checkmark & \checkmark & \checkmark & \checkmark & \checkmark & \checkmark \\
%     & inter   & \checkmark &   &   &   & \checkmark & \checkmark \\
%     %\rowcolor{gray!15}
%     & intra   & \checkmark &   &   &   &   &   \\
%     \midrule
%     \multirow{2}{*}{\STAB{\rotatebox[origin=c]{90}{SE}}}
%     & module  &   & \checkmark & \checkmark & \checkmark & \checkmark & \checkmark \\
%     %\rowcolor{gray!15}
%     & ports   &   &   &   &   & \checkmark & \checkmark \\
%     \bottomrule
% \end{tabular}

% \newcolumntype{g}{>{\columncolor{gray!15}}c}
% \begin{tabular}{cl|gc|gcgc|gc|g}
%   \toprule
%   \rowcolor{white}
%   & & \multicolumn{2}{c|}{\textit{Coding}} & \multicolumn{4}{c|}{\textit{Software configuration}} & \multicolumn{2}{c|}{\textit{Provisioning}} & \textit{Orchestration} \\
%   \midrule
%   & & \shell & \fractal & \textbf{\ansible} & \puppet & \deployware & \textbf{\aeolus} & \juju & \tosca & \kubernetes \\
%   & & & \cite{} & \cite{} & \cite{} & \cite{} & \cite{} & \cite{} & \cite{} & \cite{} \\ 
%   \midrule
%   \multirow{4}{*}{\STAB{\rotatebox[origin=c]{90}{Soft. eng.}}}
%   & components & M & \checkmark & \checkmark & \checkmark & \checkmark & \checkmark & \checkmark & \checkmark & \checkmark \\
%   & tasks & M & M & \checkmark & M & (\checkmark) & \checkmark & \checkmark & M & \\
%   & resources & M & M & & \checkmark & & & & \checkmark & \\
%   & sep. of con. & M & (\checkmark) & (\checkmark) & \checkmark & \checkmark & (\checkmark) & (\checkmark) & \checkmark & \checkmark \\
%   \midrule
%   \multirow{5}{*}{\STAB{\rotatebox[origin=c]{90}{Parallelism}}}
%   & SIMH & M & (\checkmark) & \checkmark & \checkmark & (\checkmark) & \checkmark & \checkmark & (\checkmark)& \checkmark\\
%   & inter-comp & M & (\checkmark) & & & (\checkmark) & \checkmark & \checkmark & (\checkmark)& (\checkmark)\\
%   & inter-task & M & M & & & & \checkmark & \checkmark & & \\
%   & inter-resource & M & M & & (\checkmark) & & & & (\checkmark) & \\
%   & intra-task & M & M & M & M & M & M & M & M & \\
%   \midrule
%   & formal & & \checkmark & & & (\checkmark) & \checkmark & & \checkmark & \\
%   \midrule
%   & score & 9 & 17 & 11 & 14 & 14 & 20 & 17 & 18 & 11\\
%     \bottomrule
% \end{tabular}

% without resources and puppet

\newcolumntype{g}{>{\columncolor{gray!15}}c}
\begin{tabular}{cl|gc|gcg|cg|c}
  \toprule
  \rowcolor{white}
  & & \multicolumn{2}{c|}{\textit{Coding}} & \multicolumn{3}{c|}{\textit{Software configuration}} & \multicolumn{2}{c|}{\textit{Infrastructure}} & \textit{Orchestration} \\
  \midrule
  & & \shell & \fractal & \deployware & \ansible & \aeolus & \juju & \tosca & \kubernetes \\
  & & & \cite{Baude,Blair2009,baude:hal-01001043} & \cite{flissi2008ccgrid} & \cite{ansible:web} & \cite{dicosmo:hal-01233489,dicosmo2014ic,zwolakowski:tel-01172022} & \cite{juju:web} & \cite{tosca:web,brogi2018,7561358,8599581} & \cite{kubernetes:web,43826} \\ 
  \midrule
  \multirow{3}{*}{\STAB{\rotatebox[origin=c]{90}{SE}}}
  & components & M & \checkmark & \checkmark & \checkmark & \checkmark & \checkmark & \checkmark & \checkmark \\
  & tasks & M & M & (\checkmark) & \checkmark & \checkmark & (\checkmark) & M & - \\
  & sep. of con. & M & \checkmark & \checkmark & (\checkmark) & (\checkmark) & (\checkmark) & \checkmark & \checkmark \\
  \midrule
  \multirow{4}{*}{\STAB{\rotatebox[origin=c]{90}{Parallel}}}
  & SIMH & M & \checkmark & (\checkmark) & \checkmark & \checkmark & \checkmark & (\checkmark)& \checkmark\\
  & inter-comp & M & \checkmark & (\checkmark) & - & \checkmark & \checkmark & (\checkmark)& (\checkmark)\\
  & inter-comp-tasks & M & M & - & - & \checkmark & (\checkmark) & - & - \\
  & intra-comp-tasks & M & M & M & M & M & M & M & - \\
  \midrule
  & formal & - & \checkmark & (\checkmark) & - & \checkmark & - & \checkmark & -\\
  \midrule
  & score & 7 & 18 & 15 & 12 & 21 & 16 & 15 & 11\\
    \bottomrule
\end{tabular}



  \caption{Comparison of commissioning solutions based on aspects
  regarding parallelism (performance) and software engineering (SE).}
  \label{tab:comparison}
\end{table}

\paragraph{Performance}
From our study of distributed software commissioning tools, we define three
levels regarding \emph{performance} which reflect the different degrees in
the expressiveness of parallelism observed in the previous tools.
The first level corresponds to the ability to map a set of tasks to multiple
machines. In such case, at runtime, the same tasks are executed in parallel on
different nodes. We call this level: \emph{node parallelism}. For instance, this
is typically the way parallelism is managed in \ansible (also applied in
\deployware and \kubernetes).
\emph{Inter-component parallelism} corresponds to the ability to express
dependencies between tasks of different components. At runtime, both components
are commissioned in parallel as long as no dependency is reached between them.
\aeolus and \juju are such examples of tools providing this level of
parallelism. Finally, \emph{intra-component parallelism} reflects the ability to
express task parallelism in the component internals, which can be run in
parallel when the component is instantiated. This level is hard to achieve with
\shell scrips, and is missing from the other tools.
The more levels a tool has, the more efficient it is.

\paragraph{Software engineering}
The building blocks to improve separation of concerns, code re-use and
maintainability are software modules or components, and ports.
There is no framework in \shell scripts to define modules and ports. \ansible,
\deployware and \kubernetes feature modules but no ports so either the
commissioning is error-prone (\ie \ansible and \kubernetes) or the tool has to
rely on the notion of dependency to offer limited coordination between modules.
To tackle this issue, \juju and \aeolus define ports. Ports enable developer to
provide enough information from a component to avoid inspecting its internals
and being able to manage coordination between them. As depicted
in~\cref{tab:comparison}, the solutions that provide ports are enable to manage
inter-module parallelism.

%----------------------------
\subsection{Discussions}
%----------------------------

One further aspect of the level of programmability provided by commissioning
tools. When the tool is highly programmable, developers have the ability to
manage their own code organization. This is an important aspect to finely
express what can be executed in parallel (thus it impacts performance). For
instance \deployware and \juju limit the internals of their modules to a
predefined set of actions (eg config, install) while such actions might contain
instructions that can be declared parallelable.
%
Our solution differs by letting developers abstract a set of instructions in the
form of transitions. Transitions can be expressed in parallel, and their
synchronisation is declared using states.
%high programmable can also  improve maintainability by isolating specfic code
%in a transition.

Finally, a desired feature for distributed software commissioning is automation.
Typically an operator should download the desired modules to compose a
distributed software, instantiate and connect them through their ports, then run
the commissioning process. This aspect requires an operational semantic to
express how the commissioning process automatically progresses with respect to
inter and intra-component dependencies. For instance, \aeolus, which supports
only inter-component parallelism, is a component-based model that relies on
states and transitions to manage inter-component dependencies. However, since
the model lacks an operational semantic, each operation is planned and executed
by an external scheduler. In this work, one of our contribution is to provide a
operational semantic which drives the execution of the distributed software
commissioning, and will be explained in \cref{subsec:operational_semantics}.

As depicted in~\cref{tab:comparison}, we can conclude from this analysis that
individually, none of the above attempts adequately provide a commissioning tool
which can express both a high-degree of parallelism and feature software
engineering aspects like separation of concerns.
In the rest of this paper, we define \mad, our contribution which is inspired by
\aeolus as a component-based model and relies on modules and ports to express all
the parallelism levels described above. In the rest of the paper, we will
focus the comparison of our solution with \ansible and \aeolus, since the former
is widely used in production, while the second is a research effort that
provides most of the above features.

