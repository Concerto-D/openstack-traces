% 1. introduce a bit the context (maybe in the introduction)
% - languages for dist. soft. commissioning
% - efficiency of dist. soft. commisioning
% - selected subset of RW and why

%--------
\subsection{Distributed software commissioning}

A distributed software is a software designed in a modular fashion
where each component (or module or service) is responsible for a
specific part of the overall objective, and will collaborate at
runtime with other components that are potentially hosted on distant
nodes (\ie machines) in the network.
%
The commissioning procedure of a distributed software is the procedure
responsible for leading the set of components in a valid running state
while guaranteeing their correct configurations. Thus, distributed
software commissioning is somewhere in between the components
(functional part) and the running distributed software on the
infrastructure (management part). As a result the particularity of
distributed software commissioning is to need information both from
the components behaviors and from the underlying infrastructure on
which each components will be executed.
% who are the actors of commissionings?
For this reason multiple human actors are often involved in the
commissioning procedure: developers and system administrators or
operators. This frontier is often called the DevOps domain where the
commissioning fits in.

First, each component is written by a given \emph{developer}, and this
developer often also codes a set of \emph{control interfaces} to act
on the component. Examples of control interfaces could be
\texttt{start}, \texttt{stop} or \texttt{update} etc. Such control
interfaces will take part in the component commissioning. However,
they are not sufficient as they only serve the component level, not
the infrastructure level. Thus, in addition \emph{configuration files}
are often used as \emph{infrastructure interfaces} to get information
from sysadmins, operators or other automatic tools. Third, component
often have a set of packages or libraries requirements to work
properly. Those requirements are directly related to the
\emph{infrastructure management} as some of them may already be
installed on machines. In the end, the overall commissioning procedure
of a single software component is complex as it juggles with the
control interfaces, infrastructure interfaces and infrastructure
management. More precisely the commissioning procedure of a single
component is a \emph{coordination program} between different DevOps
interfaces. Most of the time such procedure are explained in a
\texttt{README} file, in a documentation or by an online
webpage\footnote{apache red hat}.

When considering a complete distributed software composed of a set of
components, the picture becomes even darker as it is needed to
coordinate the commissioning procedure of all the components together
with their associated interactions. For instance, when installing the
very basic Apache/MariaDB an additional webpage to combine the Apache
component with the MariaDB component is needed\footnote{red
  hat}. Another example could be the commissioning of Spark on top of
Yarn\footnote{spark on Yarn}.

\HC[Dim]{ce listing ne va sans doute pas ici (trop tôt ou top détaillé) mais je le laisse}
Many interesting challenges arise from the DevOps community and from
distributed software commissioning. The challenges studied in this
paper are the following:
\begin{itemize}
\item enhancing the automation of distributed software commissionings;
\item enhancing the separation of concerns (software engineering)
  between developers and operators in distributed software
  commissionings;
\item enhancing and adapting composition mechanisms of software
  engineering for distributed software commissionings;
\item enhancing the efficiency of distributed software commissionings;
\item promoting formal models and formal methods to have guarantees on
  distributed software commissionings.
\end{itemize}

%--------
\subsection{Automation of distributed software commissioning}
% 1. scripting
% 2. infrastructure as code
% 3. provisioning tools
% 4. orchestration tools
% 5. scope of the paper
% - software commissioning of a single distributed software
% - generic solution that could use or not provisioning tools
% - scripting/IaC but keep Kubernetes and Juju

Four classes of tools and languages can be considered to automate and
enhance software engineering properties of commissioning procedures.

\paragraph{Scripting languages}
First, one usual way of automating commissioning procedures is to use
\emph{scripting languages} such as \shell or \ruby for instance. Such
languages are very flexible and well known by system administrators
and operators but suffer from many limitations, particularly from a
software engineering viewpoint. In particular, they are not well
suited to deal with separation of concerns, composition and complex
coordination mechanisms.

\paragraph{Software configuration tools}
A second class of languages and tools is called \emph{software
  configuration tools}. This class contains the set of quite recent
DevOps tools such as \ansible~\cite{}, \deployware~\cite{},
\puppet~\cite{}, \chef~\cite{}, \salt~\cite{}, \aeolus~\cite{} etc. The
goal of these tools is to enhance the productivity when defining
components or services commissionings on machines and the coordination
of their tasks. Most of the time these tools add an abstraction on top
of scripting languages to hide some of their technical details (\eg
SSH connections). These tools use different methods to acheive
productivity. For instance, Ansible adopts an imperative style in
\texttt{yaml} as a series of tasks to execute, while Puppet adopts a
declarative approach in \texttt{Python} by combining different
instances of resources (\ie service, packages etc.).

\paragraph{Infrastructure definition tools}
Some tools, that are sometimes called \emph{infrastructure definition
  tools}~\cite{}, or \emph{provisioning tools}, have been specifically
designed to be able to handle complex distributed infrastructures
composed of multiple clusters and multiple machines shared between
users. Managing such infrastructure is very difficult and error prone
as each application and each user may possibly need different
requirements, different operating systems etc. Virtualization has
partly been introduced to enhance infrastructure management (in
addition to enhancing portability of applications, isolation between
users etc.). This kind of tools, that counts \docker~\cite{},
\terraform~\cite{}, \juju~\cite{}, \cloudformation~\cite{} or
\heat~\cite{} for instance, use virtualization mechanisms (\eg virtual
machines, containers) to reduce the commissioning process to a set of
commands to ask for resources to deploy an image on a machine, thus
eventually reducing the complexity of the commissioning procedures if
the needed image is already available. Some of these tools are
specific to a given virtualization technique (\eg \docker,
\cloudformation and \heat), while others offer \emph{providers} for
different kinds of Cloud infrastructures (\eg \terraform, \juju).

\paragraph{Orchestration tools}
Finally, some recent tools go further by offering an
\emph{orchestration} level to handle complex shared set of machines
running many services simultaneously. In such case being able to
deploy or install distributed software is only a part of the problem
as services also need to be restarted in case of fault, or to be
scaled in case of overload etc. The famous \kubernetes~\cite{} enter
in this class as well as \dockerswarm~\cite{} or some implementation
of \tosca~\cite{} for instance. If these tools are above the scope of
this paper, a subpart of their architecture handle distributed
software commissionings. For instance \kubernetes relies on \docker
for commissionings while TOSCA relies on \emph{artifacts} concepts.

In pratice, these tools are often combined by system administrators
and operators to acheive both distributed software commissioning and
infrastructure management. \emph{This paper specifically focuses on
enhancing \emph{software configuration tools} by improving separation
of concerns, composition and efficiency.} In
Section~\ref{sec:related_work}, though, will compare in details tools
from the four classes: Bash, Ansible, Deployware, Aeolus, Juju and
Kubernetes.

%--------
\subsection{Efficiency of distributed software commissioning}
% 1. Optimize the commissioning by choosing adequate commands and
% tools
% 2. if using VM or Docker images: improve the boot time (Yolo)
% 3. if using Docker images: optimize placement of chunks to reduce
% the download time (Nitro)
% 4. increase the level of parallelism of commissionings
% 5. scope of the paper = parallelism

Improving the efficiency of distributed software commissioning may
seem useless as such procedure is executed once and forall. This claim
though does not consider the problem as a whole.
%
First, the commissioning procedure of complex distributed software
such as OpenStack (Section~\ref{sec:use-cases}) can take from 15
minutes to more than one hour to finish, thus gaining in speed is
significant. Second, a system administrator do not perform the
commissioning process only once. When new machines or new clusters are
installed in her infrastructure, or when updates are needed
commissioning procedures need to be executed again. Third, when
designing software commissioning, errors may often occur leading to
re-execution of the procedures multiple times.
% reconfiguration
In addition to this, having an efficient commissioning procedure is a
first step toward efficient reconfigurations of dynamic software
systems (\eg self-adaptive systems) and dynamic infrastructures (\eg
Edge Computing) where disruption time of services must be minimized.
% experiments and continuous integration
Furthermore, and as explained in the introduction, there are two
specific contexts where distributed software commissionings are
extensively executed: research or industry \emph{experiments} and
\emph{continuous integration} of distributed software systems. For
instance, by exploring the traces of the OpenStack CI platform we have
observed that the OpenStack commissioning is called around 6,000 times
in 24 hours.

Many different techniques could be studied to improve the efficiency
of distributed software commissionings. A few of them are given below:
\begin{itemize}
\item by using optimized and adapted system commands in commissioning
  scripts and program (\eg \texttt{rsync} instead of many
  \texttt{scp});
\item by working on the optimization of a given system command (\eg
  \nist package manager instead of \texttt{apt-get});
\item if using virtual images or container images, by improving the
  boot time of hypervisors~\cite{yolo};
\item if using \docker images, by optimizing the placement of image
  layers on the network~\cite{Nitro};
\item by exposing more parallelism in commissioning
  languages~\cite{madeus, aeolus}.
\end{itemize}

\emph{The scope of this paper is to study how to enhance the last point of
the above items.}