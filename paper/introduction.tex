%This paper focuses on one specific challenge related to distributed
%software deployment: \emph{distributed software commissioning}. By
%software commissioning we mean the complete installation,
%configuration and testing process when deploying distributed software
%on physical distributed resources, with or without a virtualization
%layer in between. This process is complex and error-prone because of
%the specificity of the installation process according to the operating
%system, the different kinds of virtualization layers used between the
%physical machines and the pieces of software, the amount of possible
%configuration options~\cite{Xu:2015:SAT:2775083.2791577}. Recently,
%commissioning (or configuration) management tools such as
%Ansible\footnote{\url{https://www.ansible.com/}}, or
%Puppet\footnote{\url{https://puppet.com}}, have been widely adopted by
%system operators. These tools commonly include good
%software-engineering practices such as code reuse and composition in
%management and configuration scripts. It is nowadays possible to build
%a new installation by assembling different pieces of existing
%installations\footnote{\url{https://galaxy.ansible.com/}}\footnote{\url{https://forge.puppet.com/}}
%which improves the productivity of system operators and prevents many
%errors. Many distributed software commissioning are nowadays written
%with one the three above tools and by using containers between the
%host operating system and the pieces of software, such that
%portability of installations is improved. For instance, OpenStack,
%which is the de-facto open source operating system of Cloud
%infrastructures, can be automatically installed on clusters by using
%the
%\href{https://docs.openstack.org/kolla-ansible/latest/}{\emph{kolla-ansible}
%  project}, which uses both Docker containers and Ansible.
%
%Yet, even for such well-established software, there is still much room
%for improving the efficiency of the commissioning process (\ie
%reducing deployment times, minimizing services interruptions etc.).


%correction: software is uncountable
Distributed software is software designed in a modular fashion
where each module, component, or service is responsible for a
specific part of the overall objective, and collaborates at runtime
with other components that are potentially hosted on distant nodes
(\ie machines) in the network. In the rest of this paper we will
use the generic term \emph{component} to refer
indifferently to a module or a service.
%
What is called distributed software commissioning in this paper could
also be called \emph{deployment}. However, we have noticed that,
according to the community, a different meaning could be given to the
deployment wording. For this reason, we have deliberately chosen an
unusual term that we clearly define hereafter.
%
The commissioning procedure of a distributed software is the procedure
responsible for leading the set of components to a valid running state
while guaranteeing correct configurations and interactions. Thus,
distributed software commissioning is somewhere between the
components' development (functional part) and the running state of the
distributed software on the infrastructure (management part). As a
result, the particularity of a distributed software commissioning is
that it needs information both from the components behaviors and from
the underlying infrastructure on which each component will be
executed. For this reason, multiple human actors are often involved in
the commissioning procedure: developers, system administrators and
system operators. This frontier is often called the DevOps domain
where the commissioning fits in. Note that our definition of software
commissioning does not include dynamic reconfiguration aspects of the
distributed software management that is left to future work.

When designing distributed software, first, the code of each component
is written by a given \emph{developer}, and this developer often also
codes a set of \emph{control interfaces} to act on the
component. Examples of control interfaces could be \texttt{start},
\texttt{stop} or \texttt{update}, for instance. Most of the time such
control interfaces take part in the component commissioning. However,
they are not sufficient as they only serve the component level, not
the infrastructure level, nor the configuration
level. \emph{Configuration files} are often used additionally as
\emph{configuration interfaces} to get information from system
administrators or operators that cannot be known in advance when
writing the functional code of the component. Thirdly, components
often require a set of packages or libraries to be installed on the
host operating system to work properly. Those requirements are
directly related to the \emph{infrastructure management} as some of
them may already be installed on machines.
%In the end, the overall commissioning procedure
%of a single software component is complex because it juggles with the
%control interfaces, infrastructure interfaces and infrastructure
%management information. In other words,
In the end, the commissioning procedure of a single component is a
\emph{coordination program} between different DevOps
interfaces. Commissioning procedures are often explained in
\texttt{README} files, or in documentation\footnote{apache red hat}.
%
% correction: documentations (plural) is very uncommon, the
% uncountable form is preferred
%
When considering a complete distributed software composed of a set of
components, the picture becomes even more complex as it is needed to
coordinate the commissioning procedure of all the components together
with their associated interactions. For instance, when installing a
very basic Apache/MariaDB system, additional documentation is
required\footnote{red hat} to combine the Apache component with the
MariaDB component. Another example could be the commissioning of Spark
on top of Yarn\footnote{spark on Yarn}.

%ne pas mettre sous cette forme
%\HC[Dim]{ce listing ne va sans doute pas ici (trop tôt ou top détaillé) mais je le laisse}
%Many interesting challenges arise from the DevOps community and from
%distributed software commissioning. The challenges studied in this
%paper are the following:
%\begin{itemize}
%\item enhancing the automation of distributed software commissionings;
%\item enhancing the separation of concerns between developers and
%  operators in distributed software commissionings;
%\item enhancing and adapting composition mechanisms in software
%  commissionings;
%\item enhancing the efficiency of distributed software commissionings;
%\item promoting formal models and formal methods to have guarantees on
%  distributed software commissionings.
%\end{itemize}

%%% the three important metrics of this contribution
In this paper three metrics are considered important regarding the
quality of the automation of distributed software commissioning: (1) a
clear separation of concerns between the different actors of the
commissioning procedure (\ie developers and system administrators) to
enhance productvity, maintainability and reusability of commissioning
codes; (2) the efficiency of the commissioning procedure in terms of
parallelism expressivity; and (3) the formalization of the tool such
that formal properties can be proven on a given commissioning
procedure. The contribution of this paper improves the related work by
suceeding in the combination of these three metrics.

%%% precision on the efficiency
Improving the efficiency of distributed software commissioning may
seem useless as this procedure is executed once and for all. However,
this claim does not consider the problem as a whole and need
clarifications.
%
First, the commissioning procedure of complex distributed software
such as OpenStack (Section~\ref{sec:use-cases}) can take from 15
minutes to more than one hour to finish, thus gaining in speed is
significant. Second, a system administrator does not perform the
commissioning process only once. When new machines or new clusters are
installed in their infrastructure, when errors occur, or when updates
are needed, commissioning procedures are executed again. In addition
to this, having an efficient commissioning procedure is a first step
toward efficient reconfigurations of dynamic software systems (\eg
self-adaptive systems) and dynamic infrastructures (\eg edge
computing) where disruption time of services must be minimized.
% experiments and continuous integration
Furthermore, there are two specific contexts where distributed
software commissionings are extensively executed: research or industry
\emph{experiments} and \emph{continuous integration} of distributed
software systems. For instance, by exploring the traces of the
OpenStack CI platform, we have observed that the OpenStack
commissioning is called around 6,000 times in 24 hours.

Many different techniques could be studied to improve the efficiency
of distributed software commissionings such as for instance: using
optimized and adapted system commands in commissioning scripts and
programs (\eg \texttt{rsync} instead of many \texttt{scp}); working on
the optimization of a given system command (\eg \nix package manager
instead of \texttt{apt-get}); if using virtual images or container
images, improving the boot time of
hypervisors~\cite{nguyen:hal-02172288}; if using \docker images,
optimizing the placement of image layers on the
network~\cite{darrous:hal-01745405}; exposing more parallelism in
commissioning languages~\cite{dicosmo:hal-01233489}. The scope of this
paper is to study this last point, in other words how to improve the
parallelism expressivity of commissioning tools and languages.

%%% automation of commissioning
% Plus de détails ?
%The automation of commissioning procedures has largely been
%studied. First, industry and opensource DevOps communities have
%contributed to the birth of a broad set of languages and tools such as
%\ansible, \puppet or \kubernetes among many others. Second, academic
%initiatives have lead to models and framework with interesting
%software engineering properties and formalizations. However, none of
%these contributions has suceeded in combining the three above
%aspects.

In~\cite{chardet:hal-01858150} we have presented a formal model for
distributed software commissioning, namely \mad, that theoretically
improves the parallelism level exposed to the user. This journal paper
brings the following additional contributions compared to our previous
publication:
\begin{itemize}
\item an extended study of the related work;
\item a revised and streamlined \mad formal model that offers
  strongers reachability guarantees;
\item a theoretical performance prediction model of \mad;
\item a concrete language, a prototype and an extended reproducible
  evaluation on both synthetic use-cases and a large real use-case
  on real infrastructures.
\end{itemize}

%%% outline
The rest of this paper is organized as
follows. Section~\ref{sec:related_work} studies the related
work. Section~\ref{sec:mad_model} gives an overview of \mad as well as
its concrete language. Sections~\ref{sec:formal_model}
and~\ref{sec:perf_model} respectively detail the formalization of \mad
as well as its associated performance prediction
model. Sections~\ref{sec:evaluations} and~\ref{sec:use-cases}
respectively present an evaluation of the \mad prototype on synthetic
use-cases and an evaluation of the contribution interests on the real
use-case of OpenStack commissioning.  Finally, Section~\ref{sec:conc}
concludes and gives some perspectives.

