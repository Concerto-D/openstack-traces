%\HC[Dim]{Attention copier/coller iFM, a retravailler}

%This paper focuses on one specific challenge related to distributed
%software deployment: \emph{distributed software commissioning}. By
%software commissioning we mean the complete installation,
%configuration and testing process when deploying distributed software
%on physical distributed resources, with or without a virtualization
%layer in between. This process is complex and error-prone because of
%the specificity of the installation process according to the operating
%system, the different kinds of virtualization layers used between the
%physical machines and the pieces of software, the amount of possible
%configuration options~\cite{Xu:2015:SAT:2775083.2791577}. Recently,
%commissioning (or configuration) management tools such as
%Ansible\footnote{\url{https://www.ansible.com/}}, or
%Puppet\footnote{\url{https://puppet.com}}, have been widely adopted by
%system operators. These tools commonly include good
%software-engineering practices such as code reuse and composition in
%management and configuration scripts. It is nowadays possible to build
%a new installation by assembling different pieces of existing
%installations\footnote{\url{https://galaxy.ansible.com/}}\footnote{\url{https://forge.puppet.com/}}
%which improves the productivity of system operators and prevents many
%errors. Many distributed software commissioning are nowadays written
%with one the three above tools and by using containers between the
%host operating system and the pieces of software, such that
%portability of installations is improved. For instance, OpenStack,
%which is the de-facto open source operating system of Cloud
%infrastructures, can be automatically installed on clusters by using
%the
%\href{https://docs.openstack.org/kolla-ansible/latest/}{\emph{kolla-ansible}
%  project}, which uses both Docker containers and Ansible.
%
%Yet, even for such well-established software, there is still much room
%for improving the efficiency of the commissioning process (\ie
%reducing deployment times, minimizing services interruptions etc.).

\HC[Dim]{Bien travailler discours sur l'interet de l'efficacite
  notamment pour continous integration et experimentations}

%As manually coding parallelism into commissioning procedures is
%technically difficult and error prone, automated parallelism
%techniques should be introduced.

% short version of the overview
\begin{enumerate}
\item what is software commissioning?
\item why software engineering in commissionings?
\item why efficiency in commissionings?
\item lack of the related work in combining both separation of
concerns and high flexibility/parallelism
\item previous paper 4pad
\item contributions of this paper
\begin{itemize}
\item extended related work
\item revised formal model
\item performance model
\item concrete language, prototype and extended evaluation of
synthetic and real use-case
\end{itemize}
\item reproducible
\item outline
\end{enumerate}

